%
%%%%%%%%%%%%%%%%%%%%%%%%%%%%%%%%%%%%%
%%%%%%     Logical  Model      %%%%%%
%%%%%%%%%%%%%%%%%%%%%%%%%%%%%%%%%%%%%
%

\subsection{LogicalModel}
\label{subsec:models_LogicalModel}
The \textbf{LogicalModel} is a model aimed to execute ROMs,
Codes and ExternalModels via a user provided control function. Basically, the control function
utilizes the inputs generated by RAVEN and the control logic provided by the user to determine which
model to execute.
\nb For this type of model, we currently require all models listed under \textbf{LogicalModel} should
have the same inputs and outputs from RAVEN point of view.

The specifications of a LogicalModel must be defined within the XML block
\xmlNode{LogicalModel}.
%
This XML node needs to contain the attributes:

\vspace{-5mm}
\begin{itemize}
  \itemsep0em
  \item \xmlAttr{name}, \xmlDesc{required string attribute}, user-defined name
  of this LogicalModel.
  %
  \nb As with the other objects, this is the name that can be used to refer to
  this specific entity from other input blocks in the XML.
  \item \xmlAttr{subType}, \xmlDesc{required string attribute}, must be kept
  empty.
  %
\end{itemize}
\vspace{-5mm}

Within the \xmlNode{LogicalModel} XML node, the multiple entities that constitute
this LogicalModel needs to be inputted.

\begin{itemize}
  \item \xmlNode{Model}, \xmlDesc{XML node, required parameter}.
  %
  The text portion of this node needs to contain the name of the Model
  %
  \assemblerAttrDescription{Model}
  %
  \nb The user can provided various \xmlNode{Model} entities, including \xmlString{Code},
  \xmlString{ROM} and \xmlString{ExternalModel}.

  \item \xmlNode{ControlFunction}, \xmlDesc{XML node, required parameter}.
  %
  The text portion of this node needs to contain the name of the function.
  %
  \assemblerAttrDescription{ControlFunction}
  \nb In order to work properly, this function must have a method named ``evaluate"
  that returns a single python str object representing the model that would be
  executed.
\end{itemize}

\textbf{Example (LogicalModel using external models):}
\begin{lstlisting}[style=XML,morekeywords={subType,debug,name,class,type}]
<Simulation>
  ...
  <Models>
    <ExternalModel ModuleToLoad="sum" name="sum" subType="">
      <variables>x, y, z</variables>
    </ExternalModel>

    <ExternalModel ModuleToLoad="minus" name="minus" subType="">
      <variables>x, y, z</variables>
    </ExternalModel>

    <ExternalModel ModuleToLoad="multiply" name="multiply" subType="">
      <variables>x, y, z</variables>
    </ExternalModel>

    <LogicalModel name="logical" subType="">
      <Model class="Models" type="ExternalModel">sum</Model>
      <Model class="Models" type="ExternalModel">minus</Model>
      <Model class="Models" type="ExternalModel">multiply</Model>
      <ControlFunction class="Functions" type="External">control</ControlFunction>
    </LogicalModel>
  </Models>
  ...
  <Steps>
    <MultiRun name="mc">
      <Input class="DataObjects" type="PointSet">inputHolder</Input>
      <Model class="Models" type="LogicalModel">logical</Model>
      <Sampler class="Samplers" type="MonteCarlo">MonteCarlo</Sampler>
      <Output class="DataObjects" type="PointSet">outSet</Output>
      <Output class="DataObjects" type="PointSet">tagetSet</Output>
      <Output class="OutStreams" type="Print">dumpOut</Output>
    </MultiRun>
  </Steps>
  ...
</Simulation>

\end{lstlisting}

Corresponding Python function for \xmlNode{ControlFunction}:
\begin{lstlisting}[language=python]
def evaluate(self):
  """
    Method required by RAVEN to run this as an external model.
    @ In, self, object, object to store members on
    @ Out, model, str, the name of external model that
      will be executed by hybrid model
  """
  model = None
  if self.x > 0 and self.y >1:
    model = 'sum'
  elif self.x > 0 and self.y <= 1:
    model = 'multiply'
  else:
    model = 'minus'
  return model
\end{lstlisting}

\textbf{Example (LogicalModel using codes):}
\begin{lstlisting}[style=XML,morekeywords={subType,debug,repeat,name,class,type}]
<Simulation>
  ...
  <Models>
    <Code name="poly" subType="GenericCode">
      <executable>logicalCode/poly_code.py</executable>
      <clargs arg="python" type="prepend"/>
      <clargs arg="-i" extension=".one" type="input"/>
      <fileargs arg="aux" extension=".two" type="input"/>
      <fileargs arg="output" type="output"/>
    </Code>
    <Code name="exp" subType="GenericCode">
      <executable>logicalCode/exp_code.py</executable>
      <clargs arg="python" type="prepend"/>
      <clargs arg="-i" extension=".one" type="input"/>
      <fileargs arg="aux" extension=".two" type="input"/>
      <fileargs arg="output" type="output"/>
    </Code>
    <LogicalModel name="logical" subType="">
      <Model class="Models" type="Code">poly</Model>
      <Model class="Models" type="Code">exp</Model>
      <ControlFunction class="Functions" type="External">control</ControlFunction>
    </LogicalModel>
  </Models>
  ...
  <Steps>
    <MultiRun name="logicalModelCode">
      <Input class="Files" type="">gen.one</Input>
      <Input class="Files" type="">gen.two</Input>
      <Model class="Models" type="LogicalModel">logical</Model>
      <Sampler class="Samplers" type="Stratified">LHS</Sampler>
      <Output class="DataObjects" type="PointSet">samples</Output>
      <Output class="OutStreams" type="Print">samples</Output>
    </MultiRun>
  </Steps>
  ...
</Simulation>
\end{lstlisting}

Corresponding Python function for \xmlNode{ControlFunction}:
\begin{lstlisting}[language=python]
def evaluate(self):
  """
    Method required by RAVEN to run this as an external model.
    @ In, self, object, object to store members on
    @ Out, model, str, the name of external model that
      will be executed by hybrid model
  """
  model = None
  if self.x > 0.5 and self.y > 1.5:
    model = 'poly'
  else:
    model = 'exp'

  return model
\end{lstlisting}
%
\nb For these examples, the user needs to provide all the inputs for the models listed
under \textbf{LogicalModel}, i.e. Files for the \textbf{Code} and DataObject for
the \textbf{ExternalModel} defined in the \textbf{LogicalModel}.
